\chapter{Introduction}

This document is designed to be a gentle introduction to programming
with the NASA Vision Workbench, a C++ image processing and machine
vision library.  The Vision Workbench was developed through a joint
effort of the Intelligent Robotics Group (IRG) and the Adaptive
Control and Evolvable Systems Group (ACES) within the Intelligent
Systems Division at NASA's Ames Research Center.  It is distributed
under the NASA Open Source Agreement (NOSA) version 1.3, which has
been certified by the Open Source Initiative (OSI).  A copy of this
agreement is included with every distribution of the Vision Workbench
in a file called {\tt COPYING}.

You can think of the Vision Workbench as a ``second-generation'' C/C++
image processing library.  It draws on the authors' experiences over
the past decade working with a number of ``first-generation''
libraries, such as OpenCV and VXL, as well as direct implementations
of image processing algorithms in C.  We have tried to select and
improve upon the best features of each of these approaches to image
processing, always with an eye toward our particular range of NASA
research applications.

The Vision Workbench was designed from the ground up to make it quick
and easy to produce reasonably efficient implemenations of a wide
range of image processing algorithms.  In many cases code written
using the Vision Workbench is significantly smaller and more readable
than code written using more traditional approaches.  We have found
that users who begin using the Vision Workbench to solve their image
processing problems never want to go back.  At its core is a rich set
of template-based image processing data types representing pixels,
images, and operations on those images, as well as mathematical
entities like vectors and geometric transformations and image file
I/O.  On top of this core it also provides a number of higher-level
image processing and machine vision modules, providing features
including camera geometry modelling, high-dynamic-range imaging,
interest point detection and matching, image mosaicing and blending,
and geospatial data management.

That said, the Vision Workbench is not for everyone, and in particular
it is not intended as a drop-in replacement for any existing image
processing toolkit.  It is specifically designed for image processing
in the context of machine vision, so it lacks support for things like
indexed color palettes that are more common in other areas.  It also
lacks a number of common features that the authors have simply not yet
had a need for, such as morpological operations.  If you encounter one
of these holes while using the Vision Workbench please let us know: if
it is an easy hole to fill we may be able to do so quickly, though we
cannot promise anything.  Finally, there are many application-level
algorithms, such as face recognition, that have been implemented using
other computer vision systems and are not currently provided by the
Vision Workbench.  If one of these meets your needs there is no
compelling reason to re-implement it using the Vision Workbench
instead.  On the other hand, if no existing high-level tool solves
your problem then you may well find that the Vision Workbench provides
the most productive platform for developing something new.

Since this is the first public release of the Vision Workbench, we
thought we should also provide you with some sense of the direction
the project is headed.  It is being actively developed by a small but
growing team at the NASA Ames Research Center.  A number of features
are currently being developed internally and may or may not be
released in the future, including improved mathematical optimization
capabilities, a set of Python bindings, and stereo image processing
tools.  Due to peculiarities of the NASA open-source process we cannot
provide snapshots of code that is under development and not yet
approved for public release.  If you have a specific use for features
that are under development, or in general if you have suggestions or
feature requests, please let us know.  We cannot promise anything, but
knowing our users' needs will help us prioritize our development and,
in particular, our open-source release schedule.

We hope that you enjoy using the Vision Workbench as much as we have
enjoyed developing it!  If you have any questions, suggestions,
compliments or concerns, please let us know.  Contact information is
available at the bottom of the README file included with your
distribution.
