\chapter{{\tt Cartography} Module}\label{ch:cartography-module}

[{\em Note: The documentation for this chapter is not yet complete... }]
$$
$$ 
The earliest robots in space were not planetary rovers -- they were
unmanned probes that studied the planetary bodies in our solar system
from afar.  Today there are roughly twenty extraterrestrial spacecraft
in active communication with earth (and only three planetary rovers), so
the bulk of the extraterrestrial data that we receive consists of
imagery that originated on round(ish) surfaces.  The natural thing to
do with this data is to merge it together into a map, but when doing
so we are faced with the same problem that has plagued cartographers
for hundreds of years: how does one flatten the globe?  This is the job
of the Vision Workbench Cartography module: to make maps.

Before diving into an introduction on planetary cartography, we will
point out another problem that is relatively new to Cartography.  The
amount of map data that we have collected about Earth and the other
bodies in our solar system is {\em immense}.  It is often impossible
to store an entire mapping data set in memory all at once, so
intelligent paging, caching, and storage strategies must be used in
order to make working with this data tractable.  For this reason, the
Cartography module is particularly powerful when used in conjunction
with Mosaic module (see Chapter \ref{ch:mosaic-module}) or the
upcoming Plate File module, which is designed to efficiently process
and combine extremely large data sets.

We will begin this chapter with a quick summary of the third party
libraries that are needed to compile the Vision Workbench.  We will
then describe the \verb#GeoReference# class, which creates a
relationship between pixel coordinates in an image and coordinates on
a globe.  Next, we will discuss the \verb#GeoTransform# class, which
provides a simple means of re-projecting map data.  We finish this
section with methods for reading and writing image files with embedded
geospatial metadata.

\section{Software Dependencies}

The Cartography module is currently built on top of two third party
libraries:

\begin{itemize}
\item GDAL   [ {\tt http://www.remotesensing.org/gdal/ } ]
\item Proj.4 [ {\tt http://proj.maptools.org/ }]
\end{itemize}

In order to enable the Cartography module, you must have these
libraries installed on your system before you configure and build the
Vision Workbench.  You may need to use the \verb#PKG_PATHS# directive
in the \verb#config.options# file if you install them in a
non-standard location as discussed in Chapter \ref{ch:gettingstarted}.

Once the library for the Cartography module has been built, the header
files for GDAL and Proj.4 are no longer needed, so you can rely solely
on linking in libraries when building your own application.

\section{The {\tt GeoReference} Class}

When you point at a location on a map, you probably want to know where
that location can be found in the real world.  This relationship
depends first and foremost on the familiar notion of a map's scale.
However, this relationship is also affected by a subtle, but extremely
important dependence on how the map is {\em projected}.  That is, the
image depicts a scene that sits on the surface of a spheroid.
However, the image is flat, so at best it represents a very slightly
distorted view of the surface.

One can imagine all sorts of different ways that the surface can be
warped or projected onto a flat plane (or, at the very least,
projecting onto a manifold that can be unfolded into a plane without
distorting distances and areas -- a sphere cannot be unfolded in this
way).  Generations of cartographers have struggled with this
topological challenge, and as a result they have developed many
different ways to ``un-fold'' the globe so that it can be represented
as a flat image.  Rather than attempt a description of these many
techniques here, we suggest you look at this excellent web site
describing all aspects of map projections.

\begin{verbatim}
  http://www.progonos.com/furuti/MapProj/CartIndex/cartIndex.html
\end{verbatim}

The Proj.4 manual is also recommended as a reference for the specific
map projections supported by the Vision Workbench.

Now would be a good time to take a break from reading this section of
the documentation to look over these references.  When you return, we
will dive into some code examples.

\subsection{The Datum}

A Vision Workbench \verb#GeoReference# object is composed of three items:

\begin{itemize}
\item {\bf The Datum}: Describes the approximate shape of the
  planetary body, as either a sphere or an ellipsoid.
\item {\bf The Projection}: As discussed above, this is the technique
  used to represent spherical coordinates in a flat projection.
\item {\bf The Affine Transform}: This is the geometric transformation
  between pixel coordinates in the image to coordinates in the map
  projection space.
\end{itemize}

\subsection{The Transformations}

Let's start by being explicit about the coordinate systems we will be
working with.  For images, we adopt the usual Vision Workbench
coordinate system wherein the upper left corner of the image is the
origin, the $u$ coordinate increases as you move right along the
columns of the image, and the $v$ coordinate increases as you move
down the rows.

For a planetary body, the coordinate of a point on the surface is
typically measured in longitude, latitude, and radius ($\theta, \phi,
r$) or in a geodetic format of longitude, latitude, and height
($\theta, \phi, h$). Lines of latitude are perpendicular to the axis
of rotation and are measured from the center line, the {\em equator}
(+/-90 degrees).  Lines of Longitude are vertical, passing through
both the North and South poles of the planet.  It is measured from a
vertical arc on the surface called the {\em meridian}.  We will
generally adopt an East positive frame of reference (latitude
increases to the east of the meridian 0-360 degrees).  The radius is
measured from the point to the planet's center of mass except in the
case of a flatten sphere. Thus the need for the alternative of height
measurements, which is the measured distance between a point and the
surface of a datum along the normal of the longitude, latitude
location.  Note that this coordinate system is similar but not
identical to spherical coordinates in a mathematical sense, where
``latitude'' would be measured from the North pole rather than the
equator.

Under this set of assumptions, if we have a pixel ($u,v$) in the image,
and we want to relate it to some planetary coordinates, we must do two
things. First we most convert the pixel measurement to units in the
projective domain. We call this domain the point measurement and its
units are defined by the projection. It can be in degrees in the case
of equirectangular projection or in terms of meters for UTM or
stereographic projection. We choose to not acknowledge this, hence the
rather non-descriptive name, point. Converting to point however is
simple and is just the forward transform through the affine
matrix. From here we must convert this point measurement to longitude
and latitude. This is achieved through the Proj4 library.

What you need to gather from the above information is that the
\verb#GeoReference# class converts measurements between 3 domains; pixel,
point, and longitude and latitude. The projection defines how a
spherical measurement is described on a plane \emph{(point
  space)}. The affine describes the translation and resolution of the
map by figuring how to convert pixels to the point measurement.

\subsection{Putting Things Together}

Creating GeoReferences is harder than reading them. Here’s how to read
and write a georeference from an image.

\begin{table}[htb]\begin{centering}
\begin{tabular}{|c|p{5 in}|} \hline
Function & Description \\ \hline \hline
\verb#read_georeference(geo,filename)# & Given a path \verb#filename#, this function will create and provide a reference to \verb#geo# a \verb#GeoReference#. \\ \hline
\verb#read_georeference(geo,rsrc)# & Given a disk image resource, \verb#rsrc#, this function will create and provide a reference to \verb#geo#. \\ \hline
\verb#read_georeferenced_image(im,geo,filename)# & Reads simultaneously both view \verb#im# and \verb#geo# from path \verb#filename#.\\ \hline
\verb#write_georeference(rsrc,geo)# & Writes \verb#geo# to a disk image resource, \verb#rsrc#. \\ \hline
\verb#write_gdal_image(filename,im,geo,opt)# & Writes view \verb#im# and \verb#geo# to output file, \verb#filename#, using write options \verb#opt#. \\ \hline
\end{tabular}
\caption{Reading and writing of {\tt GeoReference} Objects}
\end{centering}
\end{table}

We only support reading and writing of georeferenced images that the
GDAL library supports. We do however support read only of
georeferenced information from PDS images.

Once you have a GeoReference, you now have a powerful class that
allows conversion of measurements between pixel, point, and longitude
latitude measurements. Here’s a list of methods provided by
GeoReference that allow conversion:

\begin{table}[htb]\begin{centering}
\begin{tabular}{|c|p{5 in}|} \hline
Method & Description \\ \hline \hline
 & Single Vector Transforms \\ \hline
\verb#pixel_to_point(pix)# & Converts Vector2 \verb#pix# to Vector2 point. \\ \hline
\verb#point_to_pixel(pnt)# & Converts Vector2 \verb#pnt# to Vector2 pixel.\\ \hline
\verb#point_to_lonlat(pnt)# & Converts Vector2 \verb#pnt# to Vector2 longitude latitude.\\ \hline
\verb#lonlat_to_point(ll)# & Converts Vector2 \verb#ll# to Vector2 point.\\ \hline
\verb#pixel_to_lonlat(pix)# & Converts Vector2 \verb#pix# to Vector2 longitude latitude.\\ \hline
\verb#lonlat_to_pixel(ll)# & Converts Vector2 \verb#ll# to Vector2 pixel.\\ \hline
 & Bounding Box Transforms \\ \hline
\verb#pixel_to_point_bbox(pix)# & Converts BBox2i \verb#pix# to BBox2 point.\\ \hline
\verb#point_to_pixel_bbox(pnt)# & Converts BBox2 \verb#pnt# to BBox2i pixel.\\ \hline
\verb#point_to_lonlat_bbox(pnt)# & Converts BBox2 \verb#pnt# to BBox2 longitude latitude.\\ \hline
\verb#lonlat_to_point_bbox(ll)# & Converts BBox2 \verb#ll# to BBox2 point.\\ \hline
\verb#pixel_to_lonlat_bbox(pix)# & Converts BBox2i \verb#pix# to BBox2 longitude latitude.\\ \hline
\verb#lonlat_to_pixel_bbox(ll)# & Converts BBox2 \verb#pll# to BBox2i pixel.\\ \hline
\end{tabular}
\caption{Methods for transforming measurements in {\tt GeoReference}.}
\end{centering}
\end{table}

Say however you want to convert a geocentric measurement
\emph{(longitude, latitude, radius)} or geodetic measurement
\emph{(longitude, latitude, height)} to cartesian space? This is
provided for you in the Datum class, which is used by the
\verb#GeoReference# class. Getting access to it is as simple as
calling the \verb#datum()# method of \verb#GeoReference#. From there
you are given access to the following transforms of \verb#Datum#:

\begin{table}[htb]\begin{centering}
\begin{tabular}{|c|p{5 in}|} \hline
Method & Description \\ \hline \hline
\verb#radius(lon,lat)# & Returns double, in units of datum, the geodetic radius.\\ \hline
\verb#radius_of_curvature(lon,lat)# & Returns double, the radius of curvature in the prime vertical.\\ \hline
\verb#geocentric_latitude(lat)# & Returns double, geocentric latitude corresponding to geodetic latitude, \verb#lat#.\\ \hline
\verb#geocentric_radius(lon,lat,alt)# & Returns double, return distance from the center of the planet.\\ \hline
\verb#geodetic_to_cartesian(llh)# & Returns Vector3 in cartesian, given geodetic Vector3 \verb#llh#.\\ \hline
\verb#cartesian_to_geodetic(xyz)# & Returns Vector3 in geodetic, given cartesian Vector3 \verb#xyz#.\\ \hline
\end{tabular}
\caption{Methods for transforming measurements in {\tt Datum}.}
\end{centering}
\end{table}

Notice that we only provide cartesian transforms for geodetic
measurements. Geodetic measurements are safe for both flattened and
spherical datums alike. Geocentric to Cartesian is much simpler and
from its name you can tell that it only supports spherical datums. If
you wish to convert between geocentric measurements and cartesian, you
have two options which are listed below. Remember, these are for
geocentric measurements only!

From {\it SimplePointImageManipulation.h}:
\begin{verbatim}
  Vector3 llr(30,49,173700);
  Vector3 xyz = cartography::lon_lat_radius_to_xyz( llr );
  llr = cartography::xyz_to_lon_lat_radius( xyz );
\end{verbatim}

From {\it Datum.h}:
\begin{verbatim}
  Vector3 xyz =ç∂
    datum.geodetic_to_cartesian( geocentric -
                                 Vector3(0,0,datum.radius(geocentric[0],
                                                          geocentric[1]) ) );
\end{verbatim}

%% - The GeoReference Object
%%   - Creating 
%%     - proj4str
%%     - datum and transform
%%   - << streaming
%%   - Setting projections

\begin{table}[t]\begin{centering}
\begin{tabular}{|l|l|l|} \hline
Method & Description \\ \hline \hline
\verb#set_equirectangular()# & Equirectangular Projection \\ \hline
\verb#set_sinusoidal()# & Sinusoidal Projection \\ \hline
\verb#set_mercator()# & Mercator Projection \\ \hline
\verb#set_orthographic()# & Orthographic Projection \\ \hline
\verb#set_stereographic()# & Stereographic Projection \\ \hline
\verb#set_oblique_stereographic()# & Oblique Stereographic Projection \\ \hline
\verb#set_lambert_azimuthal()# & Lambert Azimuthal Projection \\ \hline
\verb#set_lambert_conformal()# Lambert \emph{(Conic)} Conformal Projetion \\ \hline
\verb#set_UTM()# & Universal Transverse Mercator (UTM) Projection (Earth only) \\ \hline
\end{tabular}
\caption{Currently supported {\tt GeoReference} map projections.}
\label{tbl:georeference-map-projections}
\end{centering}\end{table}

\section{Geospatial Image Processing}
\subsection{The {\tt GeoTransform} Functor}
%%     - Use to reproject maps 
%%     - Formulate a source and destination \verb#GeoReference# object
%%     - 
%%   - \verb#xyz_to_latlon()#

\section{Georeferenced File I/O}
\subsection{{\tt DiskImageResourceGDAL}}
