\chapter{Getting Started}

\section{Obtaining the Vision Workbench}

\section{Installing the Vision Workbench}

\section{A Simple Example Program}

Let's start off with a simple fully-functional example program, shown
in Listing \ref{lst:vwconvert.cc}.  You should be able to obtain an
electronic copy of this source file (as well as all the others listed
in this book) from wherever you obtained this document.  For now don't
worry about how this program works, though we hope it is fairly
self-explanatory.  Instead, just make sure that you can build and run
it successfully.  This will ensure that you have installed the Vision
Workbench properly on your computer and that you have correctly
configured your programming environment to use it.

\sourcelst{vwconvert.cc}{A simple demonstration program that can
copy image files and convert them from one file format to another.}

The program reads in an image from a source file on disk and writes it 
back out to a destination file, possibly using a different file format.  
It takes the source and destination filenames as two command-line arguments.  
For example, to convert a JPEG image called \verb#image.jpg# in the current 
directory into a PNG image you might say:
\begin{verbatim}
  vwconvert image.jpg image.png
\end{verbatim}
The details of how you invoke a command-line tool like this will of 
course vary from one operating system to another.
